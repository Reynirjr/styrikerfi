\documentclass{article}

% Encodings, page setup, paragraph formatting, font
\usepackage[top=0.9in, bottom=1in, left=1.5in, right=1.5in]{geometry}
\usepackage[icelandic]{babel}
\usepackage[T1]{fontenc}
\usepackage[sc]{mathpazo}
\usepackage[parfill]{parskip}
\usepackage{cancel}
\usepackage{comment}
% Tables and lists
\usepackage{booktabs,tabularx}
\usepackage{multirow}
\usepackage{enumerate}
\usepackage{adjustbox}
\usepackage{multicol}
\usepackage{enumitem}
\usepackage{xcolor}
% Math
\usepackage{amsmath, amsfonts, amssymb, amsthm}
\usepackage{gensymb}
% Graphics
\usepackage{graphicx}
\usepackage{forest}
\usepackage{tikz}
\usetikzlibrary{positioning, shapes, arrows.meta}
% Code environment
\usepackage{listingsutf8}
\definecolor{commentcolor}{RGB}{0, 128, 0} % Grænn
\definecolor{keywordcolor}{RGB}{0, 0, 255}   % Blár
\definecolor{stringcolor}{RGB}{163, 21, 21}      % Dökkrauður
\definecolor{numbercolor}{RGB}{128, 0, 128}      % Fjólublár
\definecolor{identifiercolor}{RGB}{0, 0, 0}      % Svartur

\def\ojoin{\setbox0=\hbox{$\bowtie$}%
  \rule[-.02ex]{.25em}{.4pt}\llap{\rule[\ht0]{.25em}{.4pt}}}
\def\leftouterjoin{\mathbin{\ojoin\mkern-5.8mu\bowtie}}
\def\rightouterjoin{\mathbin{\bowtie\mkern-5.8mu\ojoin}}
\def\fullouterjoin{\mathbin{\ojoin\mkern-5.8mu\bowtie\mkern-5.8mu\ojoin}}


\lstset{
    language=Java,
    basicstyle=\ttfamily,
    keywordstyle=\color{keywordcolor}\bfseries,
    commentstyle=\color{commentcolor},
    identifierstyle=\color{identifiercolor},
    stringstyle=\color{stringcolor},   
    showstringspaces=false,
    numbers=left,
    numberstyle=\tiny\color{gray},
    tabsize=2,
    breaklines=true,
    columns=fullflexible,
    keepspaces=true,
    inputencoding=utf8, 
    extendedchars=true,  
    literate=
        {á}{{\'a}}1
        {ð}{{\dh}}1
        {é}{{\'e}}1
        {í}{{\'i}}1
        {ó}{{\'o}}1
        {ú}{{\'u}}1
        {ý}{{\'y}}1
        {þ}{{\th}}1
        {æ}{{\ae}}1
        {ö}{{\"o}}1
        {Á}{{\'A}}1
        {Ð}{{\DH}}1
        {É}{{\'E}}1
        {Í}{{\'I}}1
        {Ó}{{\'O}}1
        {Ú}{{\'U}}1
        {Ý}{{\'Y}}1
        {Þ}{{\TH}}1
        {Æ}{{\AE}}1
        {Ö}{{\"O}}1,
}

% Restin af forskriftinni
\usepackage[pdftex,bookmarks=true,colorlinks=true,pdfauthor={Hafsteinn Einarsson},linkcolor=blue,urlcolor=blue]{hyperref}

%Custom Commands til að auðvelda mér lífið
\newcommand{\sv}{\textbf{Svar:}}
\newcommand{\bo}[1]{\textbf{#1}}
\newcommand{\enum}{\begin{enumerate}[label = \alph*.]}

% Hyphenation
\hyphenpenalty=5000
% Page and section numbering
\setcounter{secnumdepth}{-1} 
\pagenumbering{gobble}

\title{Verkefni 2}
\author{brj46 }
\date{vor 2025}

\begin{document}

\maketitle

\vspace{5em}

\begin{center}
    \includegraphics[width=0.6\textwidth]{vm.png}
\end{center}

\newpage

\section{Verkefni 3a}
\subsection{Spurning 1}
\bo{The scheduler of the VM hypervisor has just granted $\text{OS}_1$ CPU time for 20ms. While $\text{OS}_1$
is running in the granted time slot, the timer of $\text{OS}_2$ for triggering the CPU scheduler of
$\text{OS}_2$ expires. Describe two different options of how the VM hypervisor could deal with
this!}

\sv


The VM hypervisor could either: 

\begin{enumerate}
    \item Delay the handling of the interrupt from $\text{OS}_2$ until $\text{OS}_1$ has finished its time slot.
    once $\text{OS}_1$ has finished its time slot, the hypervisor can handle the interrupt from $\text{OS}_2$ and grant it CPU time.

    This option ensures that $\text{OS}_1$ gets its full time slot but $\text{OS}_2$ will experience a delay.

    \item Pause $\text{OS}_1$ and immediately switch to $\text{OS}_2$ to handle the interrupt. After $\text{OS}_2$ has finished its interrupt, the hypervisor can resume $\text{OS}_1$ 
    or continue with $\text{OS}_2$ depending on the scheduling policy.

    This option ensures that $\text{OS}_2$ gets its interrupt handled in a timely manner but $\text{OS}_1$ will experience an interruption.
\end{enumerate}


\subsection{Spurning 2}
\bo{In a virtual machine scenario, the different virtual machines are really completely separated
from each other: $\text{OS}_1$ running on $\text{VM}_1$ can neither access the main memory nor
the files1 of $\text{OS}_2$ running on $\text{VM}_2$. – Describe one possible solution that allows you
nevertheless to exchange data between different virtual machines}

\sv



A possible solution to exchange data between different virtual machines is to use a shared file system.

The hypervisor could provide a shared virtual disk or file system that both 
virtual machines can access. This shared storage would act as a medium 
for data exchange between the virtual machines.


\section{Verkefni 3b}
\subsection{Spurning 1}
\bo{While you should not download and run unknown software, in particular not OSes,
from untrusted sources, why is it not a problem to download and run the VM image for
solving this assignment?}

\sv


Well this software is provided by the university and is part of the course material. 
The main concern with running unknown software is the risk of malicious code, 
such as viruses or ransomware, which could harm your host system. 
However, since the VM image is provided by a trusted source, the risk is minimal.


\subsection{Spurning 2}
\bo{Create CPU load inside the guest OS, e.g. by opening a command line shell and running
on the command line: yes >/dev/null (stop later using Ctrl-C): describe briefly how
this CPU load affects your host system in terms of change of the CPU load as displayed
by some system monitor running on the host system (e.g. top on Linux or Mac OS
command line or the MS Windows task manager)?}


\sv


When running the command \texttt{yes >/dev/null} in the guest OS, the CPU load on the host system increased as I saw in my task manager.
but only from around $30\%$ to around $41\%$.


\subsection{Spurning 3}
\bo{Now, do the opposite by creating load on your host system: e.g., using yes >/dev/null
on Linux or Mac OS command line – for MS Windows, you can create a file that
contains the following lines, save it into, e.g., your home directory in a file of type .BAT,
and execute that file:
}
\begin{verbatim}
    @echo off
    :loop
    goto loop 
\end{verbatim}
\bo{
Describe briefly how this CPU load affects your guest system in terms of change of the
CPU load as displayed by, e.g., the command line system monitor top on the guest
system.}

\sv


The CPU usage in the guest OS increased because the host system was consuming CPU resources, leaving fewer resources available for the VM. 
When I created CPU load on the host system using the .Bat file, the host system's CPU usage increased significantly. 
As a result, the guest OS had less access to the CPU resources, causing its CPU usage to increase as well.


\subsection{Spurning 4}
\bo{Are the effects reported by the system monitors the same in the two above cases? If not, 
how can the different monitoring results be explained}

\sv


The effects are similar but not the same.
\begin{enumerate}
    \item When you create CPU load in the guest OS the host system's CPU usage 
    increases because the VM is consuming CPU resources allocated to it by the host.

    \item When you create CPU load on the host system the guest OS's CPU usage 
    increases because the host system is consuming CPU resources, 
    leaving fewer resources available for the VM.
\end{enumerate}

The differences in the monitoring results can be explained by how 
VirtualBox allocates CPU resources between the host and guest systems. 
The host system has direct control over the physical CPU and dynamically 
allocates resources to the VM. When the host system is under heavy load, 
it prioritizes its own processes, reducing the resources available to the VM.


\end{document}