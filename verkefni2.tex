\documentclass{article}

% Encodings, page setup, paragraph formatting, font
\usepackage[top=0.9in, bottom=1in, left=1.5in, right=1.5in]{geometry}
\usepackage[icelandic]{babel}
\usepackage[T1]{fontenc}
\usepackage[sc]{mathpazo}
\usepackage[parfill]{parskip}
\usepackage{cancel}
\usepackage{comment}
% Tables and lists
\usepackage{booktabs,tabularx}
\usepackage{multirow}
\usepackage{enumerate}
\usepackage{adjustbox}
\usepackage{multicol}
\usepackage{enumitem}
\usepackage{xcolor}
% Math
\usepackage{amsmath, amsfonts, amssymb, amsthm}
\usepackage{gensymb}
% Graphics
\usepackage{graphicx}
\usepackage{forest}
\usepackage{tikz}
\usetikzlibrary{positioning, shapes, arrows.meta}
% Code environment
\usepackage{listingsutf8}
\definecolor{commentcolor}{RGB}{0, 128, 0} % Grænn
\definecolor{keywordcolor}{RGB}{0, 0, 255}   % Blár
\definecolor{stringcolor}{RGB}{163, 21, 21}      % Dökkrauður
\definecolor{numbercolor}{RGB}{128, 0, 128}      % Fjólublár
\definecolor{identifiercolor}{RGB}{0, 0, 0}      % Svartur

\def\ojoin{\setbox0=\hbox{$\bowtie$}%
  \rule[-.02ex]{.25em}{.4pt}\llap{\rule[\ht0]{.25em}{.4pt}}}
\def\leftouterjoin{\mathbin{\ojoin\mkern-5.8mu\bowtie}}
\def\rightouterjoin{\mathbin{\bowtie\mkern-5.8mu\ojoin}}
\def\fullouterjoin{\mathbin{\ojoin\mkern-5.8mu\bowtie\mkern-5.8mu\ojoin}}


\lstset{
    language=Java,
    basicstyle=\ttfamily,
    keywordstyle=\color{keywordcolor}\bfseries,
    commentstyle=\color{commentcolor},
    identifierstyle=\color{identifiercolor},
    stringstyle=\color{stringcolor},   
    showstringspaces=false,
    numbers=left,
    numberstyle=\tiny\color{gray},
    tabsize=2,
    breaklines=true,
    columns=fullflexible,
    keepspaces=true,
    inputencoding=utf8, 
    extendedchars=true,  
    literate=
        {á}{{\'a}}1
        {ð}{{\dh}}1
        {é}{{\'e}}1
        {í}{{\'i}}1
        {ó}{{\'o}}1
        {ú}{{\'u}}1
        {ý}{{\'y}}1
        {þ}{{\th}}1
        {æ}{{\ae}}1
        {ö}{{\"o}}1
        {Á}{{\'A}}1
        {Ð}{{\DH}}1
        {É}{{\'E}}1
        {Í}{{\'I}}1
        {Ó}{{\'O}}1
        {Ú}{{\'U}}1
        {Ý}{{\'Y}}1
        {Þ}{{\TH}}1
        {Æ}{{\AE}}1
        {Ö}{{\"O}}1,
}

% Restin af forskriftinni
\usepackage[pdftex,bookmarks=true,colorlinks=true,pdfauthor={Hafsteinn Einarsson},linkcolor=blue,urlcolor=blue]{hyperref}

%Custom Commands til að auðvelda mér lífið
\newcommand{\sv}{\textbf{Svar:}}
\newcommand{\bo}[1]{\textbf{#1}}
\newcommand{\enum}{\begin{enumerate}[label = \alph*.]}

% Hyphenation
\hyphenpenalty=5000
% Page and section numbering
\setcounter{secnumdepth}{-1} 
\pagenumbering{gobble}

\title{Verkefni 2}
\author{brj46 }
\date{vor 2025}

\begin{document}

\maketitle


\vspace{5em}

\begin{center}
    \includegraphics[width=0.5\textwidth]{linux.png}
\end{center}

\newpage

\section{Verkefni 2}

Explain the POSIX standard
by covering briefly:

\begin{enumerate}
    \item
    \begin{itemize}
        \item[(a)] which standardisation bodies (more than one involved!) standardised POSIX,
        
        POSIX Is primarily standardized by the \bo{IEEE} and \bo{The Open Group}.
        
        \item[(b)] what is the name and/or number (e.g. “ISO 9000-2017” would the name/number
        and year) of and year of the latest standard,

        The latest standard is \bo{POSIX.1-2017} or \bo{IEEE Std 1003.1} and was published in 2018.

        \item[(c)] and what POSIX is about in general,

        POSIX defines a set of standards for compatibility among operating systems. Its specifications focus on:
        \begin{itemize}
            \item APIs
            \item Command-line shells and Utilities
            \item Common definitions and environment variables.
        \end{itemize}

    \end{itemize} 
    
    \item what each of the volumes
    
    \begin{itemize}
        \item[(a)] “Base Definitions”,
                \sv Explains fundemental concepts, definitions and headers.
        \item[(b)] “System Interfaces”,
                \sv Details C-language system calls and functions (APIs) for creating POSIX-compliant applications.
        \item[(c)] “Shell \& Utilities” 
                \sv Describes the standard shell (command interpreter) and core command-line utilities.
    \end{itemize}
    is about (you can omit the volume “Rationale”; for the other volumes, 1 or 2 sentences
    per volume are sufficient: just have a brief look into the different “volumes” listed at the
    above URL to get an idea about each volume).


    \item  Do some research on your own (and name the source you used) to give
    \begin{itemize}
        \item[(a)] a brief overview of the history of POSIX,
        
        \begin{itemize}
            \item POSIX was created because many different UNIX variants were incompatible with each other. Created in 1988 by the IEEE.
            \item The solution was POSIX which was a a source code standard so that different UNIX systems could be compatible.
            \item The name POSIX was chosen by Richard Stallman instead of IEEE-IX.
        \end{itemize}

        \item Sourced from Wikipedia on POSIX and https://thelinuxcode.com/posix-standard/
    \end{itemize}

\end{enumerate}








\end{document}