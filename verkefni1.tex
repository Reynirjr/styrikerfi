\documentclass{article}

% Encodings, page setup, paragraph formatting, font
\usepackage[top=0.9in, bottom=1in, left=1.5in, right=1.5in]{geometry}
\usepackage[icelandic]{babel}
\usepackage[T1]{fontenc}
\usepackage[sc]{mathpazo}
\usepackage[parfill]{parskip}
\usepackage{cancel}
\usepackage{comment}
% Tables and lists
\usepackage{booktabs,tabularx}
\usepackage{multirow}
\usepackage{enumerate}
\usepackage{adjustbox}
\usepackage{multicol}
\usepackage{enumitem}
\usepackage{xcolor}
% Math
\usepackage{amsmath, amsfonts, amssymb, amsthm}
\usepackage{gensymb}
% Graphics
\usepackage{graphicx}
\usepackage{forest}
\usepackage{tikz}
\usetikzlibrary{positioning, shapes, arrows.meta}
% Code environment
\usepackage{listingsutf8}
\definecolor{commentcolor}{RGB}{0, 128, 0} % Grænn
\definecolor{keywordcolor}{RGB}{0, 0, 255}   % Blár
\definecolor{stringcolor}{RGB}{163, 21, 21}      % Dökkrauður
\definecolor{numbercolor}{RGB}{128, 0, 128}      % Fjólublár
\definecolor{identifiercolor}{RGB}{0, 0, 0}      % Svartur

\def\ojoin{\setbox0=\hbox{$\bowtie$}%
  \rule[-.02ex]{.25em}{.4pt}\llap{\rule[\ht0]{.25em}{.4pt}}}
\def\leftouterjoin{\mathbin{\ojoin\mkern-5.8mu\bowtie}}
\def\rightouterjoin{\mathbin{\bowtie\mkern-5.8mu\ojoin}}
\def\fullouterjoin{\mathbin{\ojoin\mkern-5.8mu\bowtie\mkern-5.8mu\ojoin}}


\lstset{
    language=Java,
    basicstyle=\ttfamily,
    keywordstyle=\color{keywordcolor}\bfseries,
    commentstyle=\color{commentcolor},
    identifierstyle=\color{identifiercolor},
    stringstyle=\color{stringcolor},   
    showstringspaces=false,
    numbers=left,
    numberstyle=\tiny\color{gray},
    tabsize=2,
    breaklines=true,
    columns=fullflexible,
    keepspaces=true,
    inputencoding=utf8, 
    extendedchars=true,  
    literate=
        {á}{{\'a}}1
        {ð}{{\dh}}1
        {é}{{\'e}}1
        {í}{{\'i}}1
        {ó}{{\'o}}1
        {ú}{{\'u}}1
        {ý}{{\'y}}1
        {þ}{{\th}}1
        {æ}{{\ae}}1
        {ö}{{\"o}}1
        {Á}{{\'A}}1
        {Ð}{{\DH}}1
        {É}{{\'E}}1
        {Í}{{\'I}}1
        {Ó}{{\'O}}1
        {Ú}{{\'U}}1
        {Ý}{{\'Y}}1
        {Þ}{{\TH}}1
        {Æ}{{\AE}}1
        {Ö}{{\"O}}1,
}

% Restin af forskriftinni
\usepackage[pdftex,bookmarks=true,colorlinks=true,pdfauthor={Hafsteinn Einarsson},linkcolor=blue,urlcolor=blue]{hyperref}

%Custom Commands til að auðvelda mér lífið
\newcommand{\sv}{\textbf{Svar:}}
\newcommand{\bo}[1]{\textbf{#1}}
\newcommand{\enum}{\begin{enumerate}[label = \alph*.]}

% Hyphenation
\hyphenpenalty=5000
% Page and section numbering
\setcounter{secnumdepth}{-1} 
\pagenumbering{gobble}

\title{Verkefni 1}
\author{brj46 }
\date{vor 2025}

\begin{document}

\maketitle



\vspace{5em}

\begin{center}
    \includegraphics[width=0.7\textwidth]{Operating-system.png}
\end{center}

\vspace{3em}

\begin{center}
    \includegraphics[width=0.7\textwidth]{arduino.png}
\end{center}

\newpage

\section{Verkefni 1}

Discuss whether operating systems are needed at all!1 Consider for example the situation
where only one single fixed application program is executed on a hardware with no user
interface (like E.g. use Arduino as an example in an embedded system such as the software
running on the electronic control unit hardware, e.g. for motor control of a vehicle). Does
Arduino have an OS? What does a developer have to provide to run code? Is an OS needed
in this case or could the single fixed application program run on the hardware without an
operating systems?
In case you come to the conclusion that operating systems are not necessarily needed: why
is it nevertheless reasonable to have an operating system in-between an application program
and the hardware?

\sv

\bo{Do We Always Need an Operating System?}

In a Single fixed application there may be no need for complex task scheduling or user management.
A single program can run just on the hardware without an operating system.
Arduino is an Example of a single fixed application program that runs without a traditional OS
Instead, developers write code, which compiles into firmware and is put directly on the hardware.

\bo{Is an OS Neccesary in this Case?}

Not necessarily, A single task can be implemented without an OS.
\begin{itemize}
  \item no need for multitasking if only one application runs
  \item no user accounts or complex memory management is required
  \item The application can directly interact with the hardware registers.
\end{itemize}

\bo{Why is it Reasonable to Have an OS?}
Even if it is possible to run an application directly on hardware, an OS can provide benefits:

\begin{enumerate}
  \item \bo{Hardware Abstraction:}The OS provides libraries and drivers so you don’t have to program hardware registers and low-level details directly.
  \item \bo{Resource Management:} If you have multiple concurrent tasks or real-time constraints, an OS can handle CPU scheduling, memory allocation, and interrupts in a structured way.
  \item \bo{Portability and Reusability:} An OS offers standard APIs for example POSIX.
\end{enumerate}



\end{document}